\documentclass{article}
\usepackage[T1]{fontenc}
\usepackage{listings}
\usepackage{xcolor}
\usepackage[margin=2cm]{geometry}
\usepackage[scaled=.85]{beramono}

\lstset{
  language=C++,
  backgroundcolor=\color{black!7}, % set backgroundcolor
  basicstyle=\footnotesize\ttfamily,% basic font setting
  columns=fullflexible,
  showstringspaces=false,
}

\begin{document}

\tableofcontents

\pagebreak

%\section{Bigint}
%\lstinputlisting{../bigint.cpp}

\section{Data Structures}
\subsection{Vector}
\lstinputlisting{../vector.cpp}
\pagebreak
\subsection{Deque}
\lstinputlisting{../deque.cpp}
\pagebreak
\subsection{Priority Queue}
\lstinputlisting{../pq.cpp}
\pagebreak
\subsection{Pair}
\lstinputlisting{../pair.cpp}
\pagebreak
\subsection{Fenwick}
\lstinputlisting{../fenwick.cpp}
\pagebreak
\subsection{Segment Tree}
\lstinputlisting{../segtree.cpp}
\pagebreak
\subsection{Sparse Table}
\lstinputlisting{../sparsetable.cpp}
\pagebreak
\subsection{Binary Trie}
\lstinputlisting{../binarytrie.cpp}
\pagebreak
\subsection{DSU}
\lstinputlisting{../dsu.cpp}
\pagebreak
\subsection{AVL Tree}
\lstinputlisting{../avl.cpp}
\pagebreak

%\section{Factorial}
%\lstinputlisting{../fact.cpp}

\section{FFT}
\lstinputlisting{../fft.cpp}
\pagebreak

\section{Dijkstra}
\lstinputlisting{../dijkstra.cpp}
\pagebreak

\section{GCD}
\lstinputlisting{../gcd.cpp}
\pagebreak

\section{Grid}
\lstinputlisting{../grid.cpp}
\pagebreak

\section{Hash}
\lstinputlisting{../hash.cpp}
\pagebreak

\section{O(1) square root}
\lstinputlisting{../lib.cpp}
\pagebreak

\section{Matrix}
\lstinputlisting{../matrix.cpp}
\pagebreak

\section{Mint}
\lstinputlisting{../mint.cpp}
\pagebreak

\section{NT}
\lstinputlisting{../nt.cpp}
\pagebreak

\section{NTT}
\lstinputlisting{../ntt.cpp}
\pagebreak

%\section{Polynomial}
%\lstinputlisting{../polynomial.cpp}
%\pagebreak


%\section{Removable priority queue}
%\lstinputlisting{../removable_pq.cpp}
%\pagebreak

\section{Suffix Array}
\lstinputlisting{../suffixarray.cpp}
\pagebreak

\section{Template}
\lstinputlisting{../t.cpp}
\pagebreak


\end{document}
